\subsection{Modelo}

El modelo es la representación de los datos que maneja el software. Contiene los mecanismos y la lógica necesaria para acceder a la información y para actualizar el estado del modelo.\\

Para este proyecto, y con el objetivo de ser lo más fiel a un caso realista aplicable a una aplicación real, se ha decidido crear una \texttt{sqlite} donde se guardarán todos los datos. Así pues, nuestro modelo presenta dos elementos:

\subsubsection{Base de datos}
La base de datos consta del conjunto de claves RSA generadas, los puntos de la interpolación de newton para el tiempo estimado y el conjunto de resultados por cada operación. Técnicamente los atributos son:\\

\begin{itemize}
    \item \textbf{Newton interpolation}: Conjunto de puntos que definen el tiempo esperado para la ejecución de obtención de factores en horas siguiendo una interpolación de newton. 
    \item \textbf{RSA Keys}: Las claves privadas y públicas generadas históricamente durante las ejecuciones del programa.
    \item \textbf{Result}: Una estructura de datos que contiene un conjunto de métricas a tomar durante su ejecución. Técnicamente: \begin{itemize}
        \item \textbf{result}: El objeto del cual se han obtenido los datos.
        \item \textbf{time}: Tiempo de ejecución obtenido tras la obtención del dato anterior.
    \end{itemize}
    \item \textbf{Históricos}: Un conjunto de tablas que tienen como objetivo guardar el histórico de los resultados y tiempos para la respectiva operación:\begin{itemize}
        \item \textbf{Encrypt}: Encriptar un texto arbitrario.
        \item \textbf{Decrypt}: Desencriptar un texto encriptado.
        \item \textbf{Is Prime}: Identificar la primalidad de un número arbitrariamente grande.
        \item \textbf{Get Factors}: Identificar el conjunto de factores para un número arbitrariamente grande.
    \end{itemize}
\end{itemize}

\subsubsection{Interfaz}
Ya que el modelo es realmente la base de datos, es necesaria una interfaz que nos permita leer, modificar o escribir dentro de la base de datos. Para ello, la propia clase \texttt{Model.java} junto a \texttt{DBApi.java} nos permite ejecutar un conjunto de funciones que aplican queries a la base de datos. La mayoría de estos métodos siguen el siguiente patrón:\\

\begin{itemize}
    \item Conectar a la base de datos.
    \item Ejecutar el query.
    \item Transformar el resultado a un objeto de java (como puede ser un array).
    \item Devolver el resultado si no hay errores.
\end{itemize}\bigskip

Aunque el diseño se acerca a la posible implementación en una aplicación real, la implicación de usar queries a una base de datos nos proporciona tanto beneficios como perjuicios.\\

\textbf{Beneficios}\begin{itemize}
    \item Se delega la búsqueda y filtrado de datos a un lenguaje especialmente diseñado para ello.
    \item Los datos más usados se guardan en la caché, mientras que los demás se pueden guardar en disco.
    \item Los datos se mantienen entre ejecuciones, por lo que se puede aplicar un posterior estudio de los resultados en otros entornos y lenguajes.
\end{itemize}\bigskip

\textbf{Perjuicios}\begin{itemize}
    \item Se genera un overhead general en la aplicación proporcional a cómo esté programada la librería usada, ya que la conexión se debe abrir y cerrar en cada query que se ejecuta, además de la necesidad de transformar los resultados a un objeto del lenguaje usado.
\end{itemize}\bigskip

Aun así, tanto por interés académico como para crear ejemplos lo más cercanos a una aplicación real, se ha decidido seguir este acercamiento ante el problema de la gestión de un modelo en una aplicación.\\

Finalmente, al ser un módulo de nuestro MVC, implementa la interfaz \texttt{Service} y su método \texttt{notifyRequest} que le permite recibir notificaciones de los otros módulos del MVC.