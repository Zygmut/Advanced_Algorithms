\documentclass[12pt,journal,compsoc]{IEEEtran}
\PassOptionsToPackage{spanish}{babel}
\usepackage{comment,graphicx,graphics, multicol}
\usepackage{babel}
\usepackage{dirtytalk}
\usepackage{enumitem}
\usepackage{minted}
\usepackage[table,xcdraw,usenames,dvipsnames]{xcolor}   % Colored text

% Background color for minted and text coloring style
\definecolor{codebg}{rgb}{0.95,0.95,0.95}
\usemintedstyle{friendly}

% Code enviroment. Escape with "@"
\newenvironment{code}[2]
 {\VerbatimEnvironment
  \begin{minted}[fontsize=#1, linenos,breaklines,numbersep=8pt,gobble=0,frame=lines,bgcolor=codebg,framesep=3mm,escapeinside=@@]{#2}}
 {\end{minted}}

 
% Palatino/Palladio as is done in IEEE Computer Society journals.
% To go back to Times Roman, you can use this code:
%\renewcommand{\rmdefault}{ptm}\selectfont

% *** CITATION PACKAGES ***
%
\ifCLASSOPTIONcompsoc
  % The IEEE Computer Society needs nocompress option
  % requires cite.sty v4.0 or later (November 2003)
  \usepackage[nocompress]{cite}
\else
  % normal IEEE
  \usepackage{cite}
\fi


% *** MATH PACKAGES ***
%
\usepackage{amsmath}
% A popular package from the American Mathematical Society that provides
% many useful and powerful commands for dealing with mathematics.
%
% Note that the amsmath package sets \interdisplaylinepenalty to 10000
% thus preventing page breaks from occurring within multiline equations. Use:
%\interdisplaylinepenalty=2500
% after loading amsmath to restore such page breaks as IEEEtran.cls normally
% does. amsmath.sty is already installed on most LaTeX systems. The latest
% version and documentation can be obtained at:
% http://www.ctan.org/pkg/amsmath


% *** SPECIALIZED LIST PACKAGES ***

\usepackage{algorithmic}
% algorithmic.sty was written by Peter Williams and Rogerio Brito.
% This package provides an algorithmic environment fo describing algorithms.
% You can use the algorithmic environment in-text or within a figure
% environment to provide for a floating algorithm. Do NOT use the algorithm
% floating environment provided by algorithm.sty (by the same authors) or
% algorithm2e.sty (by Christophe Fiorio) as the IEEE does not use dedicated
% algorithm float types and packages that provide these will not provide
% correct IEEE style captions. The latest version and documentation of
% algorithmic.sty can be obtained at:
% http://www.ctan.org/pkg/algorithms
% Also of interest may be the (relatively newer and more customizable)
% algorithmicx.sty package by Szasz Janos:
% http://www.ctan.org/pkg/algorithmicx


% *** ALIGNMENT PACKAGES ***
%
\usepackage{array}
% Frank Mittelbach's and David Carlisle's array.sty patches and improves
% the standard LaTeX2e array and tabular environments to provide better
% appearance and additional user controls. As the default LaTeX2e table
% generation code is lacking to the point of almost being broken with
% respect to the quality of the end results, all users are strongly
% advised to use an enhanced (at the very least that provided by array.sty)
% set of table tools. array.sty is already installed on most systems. The
% latest version and documentation can be obtained at:
% http://www.ctan.org/pkg/array


\usepackage{eqparbox}
% Also of notable interest is Scott Pakin's eqparbox package for creating
% (automatically sized) equal width boxes - aka "natural width parboxes".
% Available at:
% http://www.ctan.org/pkg/eqparbox


% *** PDF, URL AND HYPERLINK PACKAGES ***

% NOTE: PDF hyperlink and bookmark features are not required in IEEE
%       papers and their use requires extra complexity and work.
% *** IF USING HYPERREF BE SURE AND CHANGE THE EXAMPLE PDF ***
% *** TITLE/SUBJECT/AUTHOR/KEYWORDS INFO BELOW!!           ***
\newcommand\MYhyperrefoptions{bookmarks=true,bookmarksnumbered=true,
pdfpagemode={UseOutlines},plainpages=false,pdfpagelabels=true,
colorlinks=true,linkcolor={black},citecolor={black},urlcolor={black},
pdftitle={Bare Demo of IEEEtran.cls for Computer Society Journals},%<!CHANGE!
pdfsubject={Typesetting},%<!CHANGE!
pdfauthor={Michael D. Shell},%<!CHANGE!
pdfkeywords={Computer Society, IEEEtran, journal, LaTeX, paper,
             template}}%<^!CHANGE!
\ifCLASSINFOpdf
\usepackage[\MYhyperrefoptions,pdftex]{hyperref}
\else
\usepackage[\MYhyperrefoptions,breaklinks=true,dvips]{hyperref}
\usepackage{breakurl}
\fi
% One significant drawback of using hyperref under DVI output is that the
% LaTeX compiler cannot break URLs across lines or pages as can be done
% under pdfLaTeX's PDF output via the hyperref pdftex driver. This is
% probably the single most important capability distinction between the
% DVI and PDF output. Perhaps surprisingly, all the other PDF features
% (PDF bookmarks, thumbnails, etc.) can be preserved in
% .tex->.dvi->.ps->.pdf workflow if the respective packages/scripts are
% loaded/invoked with the correct driver options (dvips, etc.). 
% As most IEEE papers use URLs sparingly (mainly in the references), this
% may not be as big an issue as with other publications.
%
% That said, Vilar Camara Neto created his breakurl.sty package which
% permits hyperref to easily break URLs even in dvi mode.
% Note that breakurl, unlike most other packages, must be loaded
% AFTER hyperref. The latest version of breakurl and its documentation can
% be obtained at:
% http://www.ctan.org/pkg/breakurl
% breakurl.sty is not for use under pdflatex pdf mode.
%
% The advanced features offer by hyperref.sty are not required for IEEE
% submission, so users should weigh these features against the added
% complexity of use.
% The package options above demonstrate how to enable PDF bookmarks
% (a type of table of contents viewable in Acrobat Reader) as well as
% PDF document information (title, subject, author and keywords) that is
% viewable in Acrobat reader's Document_Properties menu. PDF document
% information is also used extensively to automate the cataloging of PDF
% documents. The above set of options ensures that hyperlinks will not be
% colored in the text and thus will not be visible in the printed page,
% but will be active on "mouse over". USING COLORS OR OTHER HIGHLIGHTING
% OF HYPERLINKS CAN RESULT IN DOCUMENT REJECTION BY THE IEEE, especially if
% these appear on the "printed" page. IF IN DOUBT, ASK THE RELEVANT
% SUBMISSION EDITOR. You may need to add the option hypertexnames=false if
% you used duplicate equation numbers, etc., but this should not be needed
% in normal IEEE work.
% The latest version of hyperref and its documentation can be obtained at:
% http://www.ctan.org/pkg/hyperref

% correct bad hyphenation here
\hyphenation{op-tical net-works semi-conduc-tor}
\usepackage{csquotes}
\begin{document}

% PAPER HEADER 
\markboth{GEIN-3, Algoritmos avanzados, Práctica 3, \today}{}
% The only time the second header will appear is for the odd numbered pages
% after the title page when using the twoside option.
% 
% *** Note that you probably will NOT want to include the author's ***
% *** name in the headers of peer review papers.                   ***

% PAPER TITLE 
\title{Pareja de Puntos Más Cercana Mediante MVC}
% Titles are generally capitalized except for words such as a, an, and, as,
% at, but, by, for, in, nor, of, on, or, the, to and up, which are usually
% Do not put math or special symbols in the title.

% PAPER AUTHORS
\author{Alejandro Rodríguez, Rubén Palmer y Sergi Mayol}

% for Computer Society papers, we must declare the abstract and index terms
% As a general rule, do not put math, special symbols or citations in the abstract or keywords.
\IEEEtitleabstractindextext{%
\begin{abstract}
\hspace{2pt}Esta aplicación presenta una interfaz gráfica de usuario que le permite interactivamente encontrar la solución a la \say{Pareja de puntos más cercana}, para un conjunto de puntos dado y con una distribución determinada. A partir de la IU, el usuario podrá seleccionar el tipo de distribución (Gaussiana, Normal, Exponencial, ...), el número de puntos a generar, la semilla para la generación de los puntos y el tipo de algoritmo a aplicar. En esta práctica, se han implementado dos algoritmos para encontrar la solución. Uno de ellos, tiene una complejidad de $O(n^2)$ y el otro algoritmo tiene una complejidad de $O(n*logn)$.
\end{abstract}

{\scriptsize \textbf{Vídeo} - \href{https://youtu.be/CL3ixiOIs-M}{\textcolor{blue}{ver vídeo}}}
}

% make the title area
\maketitle
\IEEEdisplaynontitleabstractindextext
\IEEEpeerreviewmaketitle

% IEEEraisesectionheading removes all the unnecesary space between the first instance of the paper and the title
\IEEEraisesectionheading{\section{Introducción}\label{sec:introduction}}

% The very first letter is a 2 line initial drop letter followed
% by the rest of the first word in caps (small caps for compsoc).
% 
% form to use if the first word consists of a single letter:
% \IEEEPARstart{A}{demo} file is ....
% 
\IEEEPARstart{E}{n} este artículo explicaremos el funcionamiento y la implementación de nuestra práctica estructurando y dividiendo sus partes en los siguientes apartados:\bigskip

Inicialmente, se citarán las librerías usadas en este proyecto. Para aquellas que se hayan desarrollado específicamente, se expondrá una breve guía de su funcionamiento y uso.\bigskip

Seguidamente, se explicará cuál es la arquitectura, interfaces, funcionamiento e implementación de nuestro \textbf{M}odelo-\textbf{V}ista-\textbf{C}ontrolador (MVC) así como los métodos de mayor importancia. Para facilitar la legibilidad, se separará en cada uno de los grandes bloques de la estructura; en este caso Modelo, Vista, Controlador y Hub\bigskip

Finalmente, se mostrará una breve guía de como usar nuestra aplicación y un ejemplo de esta.

\subsection{Modelo}

El modelo es la representación de los datos que maneja el software. Contiene los mecanismos y la lógica necesaria para acceder a la información y para actualizar el estado del modelo.\\

Para este proyecto, y con el objetivo de ser lo más fiel a un caso realista aplicable a una aplicación real, se ha decidido crear una \texttt{sqlite} donde se guardarán todos los datos. Así pues, nuestro modelo presenta dos elementos:

\subsubsection{Base de datos}
La base de datos consta de una tabla por idioma y una tabla para guardar el histórico de tiempos de ejecución; Técnicamente:\\

\begin{itemize}
    \item \textbf{LANGUAGE}, donde el nombre de la tabla corresponde al nombre corto de un lenguaje. Con los atributos:\begin{itemize}
        \item \texttt{word}: Representa una palabra del lenguaje.
    \end{itemize}
    \item \textbf{Timed\_Execution}, con los atributos:\begin{itemize}
        \item \texttt{id}: Permite identificar cuál ha sido la última ejecución.
        \item \texttt{milis}: La cantidad de milisegundos usados durante la última ejecución.
    \end{itemize}
\end{itemize}

\subsubsection{Interfaz}
Ya que el modelo es realmente la base de datos, es necesaria una interfaz que nos permita leer, modificar o escribir dentro de la base de datos. Para ello, la propia clase \texttt{Model.java} nos permite ejecutar un conjunto de funciones que aplican queries a la base de datos. La mayoría de estos métodos siguen el siguiente patrón:\\

\begin{itemize}
    \item Conectar a la base de datos
    \item Ejecutar el query
    \item Transformar el resultado a un objeto de java (como puede ser un array)
    \item Devolver el resultado si no hay errores
\end{itemize}\bigskip

Aunque el diseño se acerca a la posible implementación en una aplicación real, la implicación de usar queries a una base de datos nos proporciona tanto beneficios como perjuicios.\\

\textbf{Beneficios}\begin{itemize}
    \item Se delega la búsqueda y filtrado de datos a un lenguaje especialmente diseñado para ello.
    \item Los datos más usados se guardan en la caché, mientras que los demás se pueden guardar en disco.
    \item Los datos se mantienen entre ejecuciones, por lo que se puede aplicar un posterior estudio de los resultados en otros entornos y lenguajes.
\end{itemize}\bigskip

\textbf{Perjuicios}\begin{itemize}
    \item Se genera un overhead general en la aplicación proporcional a cómo esté programada la librería usada, ya que la conexión se debe abrir y cerrar en cada query que se ejecuta, además de la necesidad de transformar los resultados a un objeto del lenguaje usado.
\end{itemize}\bigskip

Aun así, tanto por interés académico como para crear ejemplos lo más cercanos a una aplicación real, se ha decidido seguir este acercamiento ante el problema de la gestión de un modelo en una aplicación.\\

Finalmente, al ser un módulo de nuestro MVC, implementa la interfaz \texttt{Notify} y su método \texttt{notifyRequest} que le permite recibir notificaciones de los otros módulos del MVC.
\subsection{Modelo}

El modelo es la representación de los datos que maneja el software. Contiene los mecanismos y la lógica necesaria para acceder a la información y para actualizar el estado del modelo.\\

Para este proyecto, y con el objetivo de ser lo más fiel a un caso realista aplicable a una aplicación real, se ha decidido crear una \texttt{sqlite} donde se guardarán todos los datos. Así pues, nuestro modelo presenta dos elementos:

\subsubsection{Base de datos}
La base de datos consta de una tabla por idioma y una tabla para guardar el histórico de tiempos de ejecución; Técnicamente:\\

\begin{itemize}
    \item \textbf{LANGUAGE}, donde el nombre de la tabla corresponde al nombre corto de un lenguaje. Con los atributos:\begin{itemize}
        \item \texttt{word}: Representa una palabra del lenguaje.
    \end{itemize}
    \item \textbf{Timed\_Execution}, con los atributos:\begin{itemize}
        \item \texttt{id}: Permite identificar cuál ha sido la última ejecución.
        \item \texttt{milis}: La cantidad de milisegundos usados durante la última ejecución.
    \end{itemize}
\end{itemize}

\subsubsection{Interfaz}
Ya que el modelo es realmente la base de datos, es necesaria una interfaz que nos permita leer, modificar o escribir dentro de la base de datos. Para ello, la propia clase \texttt{Model.java} nos permite ejecutar un conjunto de funciones que aplican queries a la base de datos. La mayoría de estos métodos siguen el siguiente patrón:\\

\begin{itemize}
    \item Conectar a la base de datos
    \item Ejecutar el query
    \item Transformar el resultado a un objeto de java (como puede ser un array)
    \item Devolver el resultado si no hay errores
\end{itemize}\bigskip

Aunque el diseño se acerca a la posible implementación en una aplicación real, la implicación de usar queries a una base de datos nos proporciona tanto beneficios como perjuicios.\\

\textbf{Beneficios}\begin{itemize}
    \item Se delega la búsqueda y filtrado de datos a un lenguaje especialmente diseñado para ello.
    \item Los datos más usados se guardan en la caché, mientras que los demás se pueden guardar en disco.
    \item Los datos se mantienen entre ejecuciones, por lo que se puede aplicar un posterior estudio de los resultados en otros entornos y lenguajes.
\end{itemize}\bigskip

\textbf{Perjuicios}\begin{itemize}
    \item Se genera un overhead general en la aplicación proporcional a cómo esté programada la librería usada, ya que la conexión se debe abrir y cerrar en cada query que se ejecuta, además de la necesidad de transformar los resultados a un objeto del lenguaje usado.
\end{itemize}\bigskip

Aun así, tanto por interés académico como para crear ejemplos lo más cercanos a una aplicación real, se ha decidido seguir este acercamiento ante el problema de la gestión de un modelo en una aplicación.\\

Finalmente, al ser un módulo de nuestro MVC, implementa la interfaz \texttt{Notify} y su método \texttt{notifyRequest} que le permite recibir notificaciones de los otros módulos del MVC.
\subsection{Modelo}

El modelo es la representación de los datos que maneja el software. Contiene los mecanismos y la lógica necesaria para acceder a la información y para actualizar el estado del modelo.\\

Para este proyecto, y con el objetivo de ser lo más fiel a un caso realista aplicable a una aplicación real, se ha decidido crear una \texttt{sqlite} donde se guardarán todos los datos. Así pues, nuestro modelo presenta dos elementos:

\subsubsection{Base de datos}
La base de datos consta de una tabla por idioma y una tabla para guardar el histórico de tiempos de ejecución; Técnicamente:\\

\begin{itemize}
    \item \textbf{LANGUAGE}, donde el nombre de la tabla corresponde al nombre corto de un lenguaje. Con los atributos:\begin{itemize}
        \item \texttt{word}: Representa una palabra del lenguaje.
    \end{itemize}
    \item \textbf{Timed\_Execution}, con los atributos:\begin{itemize}
        \item \texttt{id}: Permite identificar cuál ha sido la última ejecución.
        \item \texttt{milis}: La cantidad de milisegundos usados durante la última ejecución.
    \end{itemize}
\end{itemize}

\subsubsection{Interfaz}
Ya que el modelo es realmente la base de datos, es necesaria una interfaz que nos permita leer, modificar o escribir dentro de la base de datos. Para ello, la propia clase \texttt{Model.java} nos permite ejecutar un conjunto de funciones que aplican queries a la base de datos. La mayoría de estos métodos siguen el siguiente patrón:\\

\begin{itemize}
    \item Conectar a la base de datos
    \item Ejecutar el query
    \item Transformar el resultado a un objeto de java (como puede ser un array)
    \item Devolver el resultado si no hay errores
\end{itemize}\bigskip

Aunque el diseño se acerca a la posible implementación en una aplicación real, la implicación de usar queries a una base de datos nos proporciona tanto beneficios como perjuicios.\\

\textbf{Beneficios}\begin{itemize}
    \item Se delega la búsqueda y filtrado de datos a un lenguaje especialmente diseñado para ello.
    \item Los datos más usados se guardan en la caché, mientras que los demás se pueden guardar en disco.
    \item Los datos se mantienen entre ejecuciones, por lo que se puede aplicar un posterior estudio de los resultados en otros entornos y lenguajes.
\end{itemize}\bigskip

\textbf{Perjuicios}\begin{itemize}
    \item Se genera un overhead general en la aplicación proporcional a cómo esté programada la librería usada, ya que la conexión se debe abrir y cerrar en cada query que se ejecuta, además de la necesidad de transformar los resultados a un objeto del lenguaje usado.
\end{itemize}\bigskip

Aun así, tanto por interés académico como para crear ejemplos lo más cercanos a una aplicación real, se ha decidido seguir este acercamiento ante el problema de la gestión de un modelo en una aplicación.\\

Finalmente, al ser un módulo de nuestro MVC, implementa la interfaz \texttt{Notify} y su método \texttt{notifyRequest} que le permite recibir notificaciones de los otros módulos del MVC.
\subsection{Modelo}

El modelo es la representación de los datos que maneja el software. Contiene los mecanismos y la lógica necesaria para acceder a la información y para actualizar el estado del modelo.\\

Para este proyecto, y con el objetivo de ser lo más fiel a un caso realista aplicable a una aplicación real, se ha decidido crear una \texttt{sqlite} donde se guardarán todos los datos. Así pues, nuestro modelo presenta dos elementos:

\subsubsection{Base de datos}
La base de datos consta de una tabla por idioma y una tabla para guardar el histórico de tiempos de ejecución; Técnicamente:\\

\begin{itemize}
    \item \textbf{LANGUAGE}, donde el nombre de la tabla corresponde al nombre corto de un lenguaje. Con los atributos:\begin{itemize}
        \item \texttt{word}: Representa una palabra del lenguaje.
    \end{itemize}
    \item \textbf{Timed\_Execution}, con los atributos:\begin{itemize}
        \item \texttt{id}: Permite identificar cuál ha sido la última ejecución.
        \item \texttt{milis}: La cantidad de milisegundos usados durante la última ejecución.
    \end{itemize}
\end{itemize}

\subsubsection{Interfaz}
Ya que el modelo es realmente la base de datos, es necesaria una interfaz que nos permita leer, modificar o escribir dentro de la base de datos. Para ello, la propia clase \texttt{Model.java} nos permite ejecutar un conjunto de funciones que aplican queries a la base de datos. La mayoría de estos métodos siguen el siguiente patrón:\\

\begin{itemize}
    \item Conectar a la base de datos
    \item Ejecutar el query
    \item Transformar el resultado a un objeto de java (como puede ser un array)
    \item Devolver el resultado si no hay errores
\end{itemize}\bigskip

Aunque el diseño se acerca a la posible implementación en una aplicación real, la implicación de usar queries a una base de datos nos proporciona tanto beneficios como perjuicios.\\

\textbf{Beneficios}\begin{itemize}
    \item Se delega la búsqueda y filtrado de datos a un lenguaje especialmente diseñado para ello.
    \item Los datos más usados se guardan en la caché, mientras que los demás se pueden guardar en disco.
    \item Los datos se mantienen entre ejecuciones, por lo que se puede aplicar un posterior estudio de los resultados en otros entornos y lenguajes.
\end{itemize}\bigskip

\textbf{Perjuicios}\begin{itemize}
    \item Se genera un overhead general en la aplicación proporcional a cómo esté programada la librería usada, ya que la conexión se debe abrir y cerrar en cada query que se ejecuta, además de la necesidad de transformar los resultados a un objeto del lenguaje usado.
\end{itemize}\bigskip

Aun así, tanto por interés académico como para crear ejemplos lo más cercanos a una aplicación real, se ha decidido seguir este acercamiento ante el problema de la gestión de un modelo en una aplicación.\\

Finalmente, al ser un módulo de nuestro MVC, implementa la interfaz \texttt{Notify} y su método \texttt{notifyRequest} que le permite recibir notificaciones de los otros módulos del MVC.

\section{Distribución del trabajo realizado}

El trabajo realizado por cada miembro de la práctica ha sido el siguiente: \bigskip

\begin{itemize}
    \item Alejandro Rodríguez:
    \begin{itemize}
        \item Desarrollador de funcionalidades extra de la IU.
    \end{itemize}
    
    \item Rubén Palmer:
    \begin{itemize}
        \item Documentación
        \item Principal desarrollador del \say{backend} de la aplicación
        \item Desarrollador general de la aplicación
        \item Desarrollador del \say{frontend}
        \item Diseñador de la implementación del patrón MVC de la aplicación
        \item Desarrollador de la plantilla base del proyecto
        \item Desarrollador de la plantilla de la documentación 
    \end{itemize}
    \item Sergi Mayol:
    \begin{itemize}
        \item Documentación
        \item Desarrollador de la librería \texttt{Better Swing}
        \item Principal desarrollador del \say{frontend} de la aplicación
        \item Desarrollador general de la aplicación
        \item Desarrollador del \say{backend} de la aplicación
        \item Diseñador de la implementación del patrón MVC de la aplicación
        \item Desarrollador de la plantilla base del proyecto
    \end{itemize}
\end{itemize}

\begin{figure}[!h]
    \centering
    \includegraphics[width=\linewidth]{alex.png}
    \caption{Contribución en el proyecto de Alejandro Rodríguez}
    \label{fig:contrib_alex}
\end{figure}

\begin{figure}[!h]
    \centering
    \includegraphics[width=\linewidth]{ruben.png}
    \caption{Contribución en el proyecto de Rubén Palmer}
    \label{fig:contrib_ruben}
\end{figure}

\begin{figure}[!h]
    \centering
    \includegraphics[width=\linewidth]{sergi.png}
    \caption{Contribución en el proyecto de Sergi Mayol}
    \label{fig:contrib_sergi}
\end{figure}

\newpage

% use section* for acknowledgment
\section*{Reconocimientos}
\begin{itemize}
    \item Se agradece la colaboración del Dr. Miquel Mascaró Portells en la resolución de dudas y supervisión del proyecto.
\end{itemize}

\begin{thebibliography}{1}

\bibitem{betterswing}
Better Swing, \emph{An easy way to develop java GUI apps}, V0.0.3. \hskip 1em plus
  0.5em minus 0.4em\relax Ver documentación completa \href{https://sergimayol.github.io/better-swing}{\textcolor{blue}{aquí}}.

\bibitem{Graphical engine}
Graphical engine, \emph{How to develop a graphical engine from scratch}. Ver enlace \href{https://www.youtube.com/watch?v=025QFeZfeyM&t=7334s}{\textcolor{blue}{aquí}}.

\bibitem{Exponential distribution}
What is the exponential distribution, what are its formulas and how does it work. Ver documentación \href{https://en.wikipedia.org/wiki/Exponential_distribution}{\textcolor{blue}{aquí}}.

\bibitem{Exponential Bernoulli}
What is the Bernoulli distribution, what are its formulas and how does it work. Ver documentación \href{https://en.wikipedia.org/wiki/Bernoulli_distribution}{\textcolor{blue}{aquí}}.

\bibitem{Exponential distribution}
What is the exponential distribution, what are its formulas and how does it work. Ver documentación \href{https://en.wikipedia.org/wiki/Normal_distribution}{\textcolor{blue}{aquí}}.

\bibitem{Divide and Conquer}
Closest pair of points problem. Ver más en \href{https://en.wikipedia.org/wiki/Closest_pair_of_points_problem}{\textcolor{blue}{aquí}}.

\end{thebibliography}

\end{document}
