\subsection{Vista}

La vista contiene los componentes para representar la interfaz del usuario (IU) del programa y las herramientas con las cuales el usuario puede interactuar con los datos de la aplicación. Adicionalmente, la vista se encarga de recibir e interpretar adecuadamente los datos obtenidos del modelo. Cabe mencionar, que al igual que el resto de componentes del MVC, la clase View implementa la interfaz \say{Service}, la cual permite la comunicación con el resto de elementos.\bigskip

\texttt{loadContent} es el encargado de cargar el contenido inicial en la ventana. Para esta práctica, carga los siguientes elementos semánticos:\bigskip

\texttt{menu}, función que crea y configura el menú de utilidades de la ventana principal. En esta, se incluyen las siguientes opciones: Abrir ventanas de estadísticas, abrir el manual de usuario, cargar la base de datos y salir del programa.\bigskip

\texttt{body}, función que crea, configura y actualiza la visualización del puzzle. Se trata de una serie de botones que permiten con un texto o imagen que representan la forma del puzzle de manera visual y agradable.\bigskip

\texttt{sidebar}, función que crea y configura la barra de opciones de la derecha de la interfaz principal. Esta permite al usuario interactuar y configurar el entorno de ejecución del algoritmo, para ello, se permite: seleccionar la imagen del puzzle, el tamaño y la heurística.\bigskip

\texttt{footer}, función que crea y configura los botones para iniciar el algoritmo, entre ellos, barajar el puzzle actual, seleccionar una semilla para barajar el puzzle y un botón para iniciar el algoritmo.\bigskip

Adicionalmente, a la vista principal, esta permite desplegar una serie de ventanas extra. Una ventana muestra, a tiempo real, el uso y consumo de la memoria de la Java virtual machine (JVM) (\ref{Stats JVM}), la otra ventana muestra las estadísticas de la ejecución de los algoritmos además de su comparación (\ref{Stats Algt}) y un manual de usuario (\ref{Manual usuario, Header}).\bigskip

Finalmente, al ser un módulo de nuestro MVC, implementa la interfaz \texttt{Service} y su método \texttt{notifyRequest} que le permite recibir notificaciones de los otros módulos del MVC.

\subsubsection{Estadísticas JVM}\label{Stats JVM}

Este apartado de la vista principal, es el encargado de enseñar a tiempo real las estadísticas de la máquina virtual de java. Concretamente, se actualiza cada 0.5 s a partir de los datos obtenidos de la clase de java \say{Runtime} y muestra la memoria libre, la memoria total y el uso de esta en una gráfica de líneas, donde el eje x es instante en el tiempo que se han obtenido los datos y el eje y su valor. Todos los datos de la memoria obtenidos están en MB.

\begin{figure}[!h]
    \centering
    \includegraphics[width=\linewidth]{MVC/View/img/stats-jvm.png}
    \caption{Interfaz estadística JVM}
    \label{fig:Ejemplo stats JVM}
\end{figure}

\subsubsection{Estadísticas de los Algoritmos}\label{Stats Algt}

Este apartado de la vista principal, es el encargado de enseñar las estadísticas de la ejecución del algoritmo de \say{Branch and Bound}. Estas estadísticas incluyen una gráfica con el tiempo de ejecución de cada ejecución del algoritmo. Los resultados (tiempos de ejecución) son representados en un gráfico de barras, donde el eje \textit{x} representa el número de ejecuciones del algoritmo y el eje y el tiempo en milisegundos que ha tardado. Dos gráficos circulares comparando los ratios de las veces que se ha referenciado, podado el árbol y cuantos se han visitado. Y finalmente, un gráfico de barras con la cantidad de movimientos realizados para encontrar la solución por heurística.

\begin{figure}[!h]
    \centering
    \includegraphics[width=\linewidth]{MVC/View/img/stats-algt.png}
    \caption{Interfaz estadísticas algortimos}
    \label{fig:Ejemplo stats Algt}
\end{figure}

Como se ha podido ver anteriormente en la imagen (\ref{fig:Ejemplo stats Algt}), se puede apreciar los diferentes datos obtenidos tras una serie de ejecuciones del algoritmo.
